%LaTeX
\synctex=1
\documentclass[a4paper,11pt,reqno]{amsart}
\usepackage[bb,wide,cthms,tikz]{gewoehn} %Custom package by the author. 
%Options: (bb, bf); (tight, wide, wider); (thms, sthms, cthms, pthms)
% \usepackage[german]{babel}
\renewcommand{\datename}{\textit{Datum}:}
\theoremstyle{uremark}
\newtheorem*{loes}{L\"osung}
\begin{document}

\title[]{L\"osung zur \"Ubung 2.25 -- Abelson und Sussman, Struktur und
    Interpretation von Computerprogrammen (SICP)} 
\author{\href{https://github.com/pzuehlke}{\ttt{https://github.com/pzuehlke}}}
% \date{\today}
\maketitle
\

\noindent \tit{\"Ubung 2.25}: Geben Sie Kombinationen von \ttt{car} und
\ttt{cdr} an, die die $ 7 $ aus jeder der folgenden Listen herauspicken:
\begin{enumerate}
    \item [(a)] \ttt{(1 3 (5 7) 9)}
    \item [(b)] \ttt{((7))} 
    \item [(c)] \ttt{(1 (2 (3 (4 (5 (6 7))))))}
\end{enumerate}

\begin{loes}\ 
\begin{enumerate}[label=\small$\bullet$]
    \item [(a)] \texttt{(car (cdr (car (cdr (cdr (liste))))))}
    \item [(b)] \texttt{(car (car liste))}
    \item [(c)] Man beachte zuerst, dass die Elemente dieser Liste nicht
        einfach die Zahlen $ 1, 2, \dots, 7 $ sind, sondern die Zahl $ 1 $ und
        eine andere Liste. Die letztere enth\"ahlt die Zahl $ 2 $ und eine
        andere Liste, usw.. Wir brauchen f\"unf Anwendungen von \texttt{cadr}
        um $ (6\ 7) $ zu erreichen, und dann noch einen \texttt{cadr} um die $
        7 $ herauszupicken:
        \begin{equation*}%\label{E:}
            \texttt{(cadr (cadr (cadr (cadr (cadr (cadr liste))))))}
        \end{equation*}
\end{enumerate}
\end{loes}

\end{document}
