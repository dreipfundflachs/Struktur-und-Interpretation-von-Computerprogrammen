%LaTeX
\synctex=1
\documentclass[a4paper,11pt,reqno]{amsart}
\usepackage[bb,wide,cthms,tikz]{gewoehn} %Custom package by the author. 
%Options: (bb, bf); (tight, wide, wider); (thms, sthms, cthms, pthms)
\usepackage[german]{babel}
\usepackage{mathtools}
\DeclarePairedDelimiter{\nint}{[}{]}
\renewcommand{\datename}{\textit{Datum}:}
\theoremstyle{uremark}
\newtheorem*{loes}{L\"osung zur \"Ubung 1.13}
\begin{document}

\title[]{L\"osung zur \"Ubung 1.13 -- Struktur und Interpretation von
Computerprogrammen (SICP) von H.~Abelson und G.\,J.~Sussman}
\author{\href{https://github.com/pzuehlke}{\ttt{https://github.com/pzuehlke}}}
% \date{\today}
\maketitle
\


\begin{loes} 
    Es sei $ x_n = \text{Fib}\,(n) $. Man beachte zuerst, dass die folgende
    zwei Gleichungen f\"ur alle $ n \ge 1 $ gelten:
    \begin{equation*}%\label{E:}
        \begin{cases}
            x_{n + 1} & = x_{n} + x_{n - 1} \\
            x_n & = x_n
        \end{cases}
    \end{equation*}
    Andererseits sind sie \"aquivalent zu der Gleichung
    \begin{equation*}%\label{E:}
        \begin{bmatrix}
            x_{n + 1} \\
            x_n
        \end{bmatrix} =\begin{bmatrix}
            1 & 1 \\
            1 & 0 
        \end{bmatrix}
        \begin{bmatrix}
            x_n \\
            x_{n - 1}
        \end{bmatrix}.
        \end{equation*}
        Sei $ A = \begin{bmatrix}
            1 & 1 \\
            1 & 0 
        \end{bmatrix} $ die Matrix an der rechten Seite der vorhergehenden
        Gleichung. Das
        charakteristiche Polynom von $ A $ ist $ \la^2 - \la - 1 $, deren
        Wurzeln
        \begin{equation*}%\label{E:}
            \phi = \frac{1 + \sqrt{5}}{2} \quad \text{und} \quad \psi = \frac{1
            - \sqrt{5}}{2}
        \end{equation*}
        sind, mit zugeh\"origen Eigenvektoren $ (\phi, 1) $, bzw. $ (\psi, 1) $
        (oder irgendwelche Vektoren, die damit kollinear sind).

        Es seien also
        \begin{equation*}%\label{E:}
            P = \begin{bmatrix}
                \phi & \psi \\
                1 & 1 
            \end{bmatrix} \quad \text{und} \quad D = \begin{bmatrix}
                \phi & 0 \\
                0 & \psi 
            \end{bmatrix}.
        \end{equation*}
        Dann gilt $ A = PDP^{-1} $, und deswegen auch $ A^n = PD^nP^{-1} $
        f\"ur alle $ n \ge 0 $. Somit gilt
        \begin{equation*}%\label{E:}
        \begin{bmatrix}
            x_{n + 1} \\
            x_n
        \end{bmatrix} = A^n
        \begin{bmatrix}
            x_1 \\
            x_{0}
        \end{bmatrix} = PD^n P^{-1}
        \begin{bmatrix}
            1 \\
            0
        \end{bmatrix}.
    \end{equation*}
    Man berechnet leicht aus dem letzten Ausdruck, dass 
    \begin{equation*}%\label{E:}
        \begin{bmatrix}
            x_{n + 1} \\
            x_n
        \end{bmatrix} = 
        \frac{1}{\sqrt{5}}\begin{bmatrix}
            \phi & \psi \\
            1 & 1 
        \end{bmatrix} \begin{bmatrix}
            \phi^n & 0 \\
            0 & \psi^n 
        \end{bmatrix} \begin{bmatrix}
            1 & -\psi \\
            -1 & \phi 
        \end{bmatrix} \begin{bmatrix}
            1 \\
            0  
        \end{bmatrix} = \frac{1}{\sqrt{5}}\begin{bmatrix}
            \phi^{n + 1} - \psi^{n + 1} \\
            \phi^n - \psi^n
        \end{bmatrix}.
    \end{equation*}
    Schlie\ss lich haben wir damit bewiesen, dass
    \begin{equation*}%\label{E:}
        \qquad \boxed{\text{Fib}(n) = x_n = \frac{1}{\sqrt{5}}\big( \phi^n -
        \psi^n \big)}\qquad (n \ge 0).
    \end{equation*}
    Au\ss erdem, weil $ \psi \approx -0,618 $, ist $ x_n = \text{Fib}\,(n) $
    die n\"achste ganze Zahl zu $ \frac{\phi^n}{\sqrt{5}} $ f\"ur alle $ n \ge
    1$ (vgl.~die Bemerkung an Seite 37).  \qed
\end{loes}

\end{document}
