%LaTeX
\synctex=1
\documentclass[a4paper,11pt,reqno]{amsart}
\usepackage[bb,wide,cthms,tikz]{gewoehn} %Custom package by the author. 
%Options: (bb, bf); (tight, wide, wider); (thms, sthms, cthms, pthms)
\usepackage[german]{babel}
\renewcommand{\datename}{\textit{Datum}:}
\theoremstyle{uremark}
\newtheorem*{loes}{L\"osung zur \"Ubung 1.1}
\begin{document}

\title[]{L\"osung zur \"Ubung 1.1 -- Struktur und Interpretation von
Computerprogrammen (SICP) von H.~Abelson und G.\,J.~Sussman}
\author{dreipfundflachs --
\href{https://github.com/dreipfundflachs}{\ttt{https://github.com/dreipfundflachs}}}
\date{\today}
\maketitle
\

\begin{loes}\ 
\begin{enumerate}[label=\small$\bullet$]
    \item \texttt{10}
    \item \texttt{12}
    \item \texttt{8}
    \item \texttt{3}
    \item \texttt{6} ($ = 8  + (-2)$)
    \item definiert $ a $, druckt aber nichts als Antwort zur\"uck
    \item definiert $ b $, druckt aber nichts als Antwort zur\"uck
    \item \texttt{19} ($ = 3 + 4 + (3 \times 4) $) 
    \item \texttt{\#f}
    \item \texttt{4} ($ = b $, denn $ b > a $ und $ b < a \times b $ sind beide
            wahr)
    \item \texttt{16} ($ = 6 + 7 + a $, denn der erste Pr\"adikat ist
        falsch, w\"ahrend der zweite wahr ist)
    \item \texttt{6} ($ = 2 + b $, denn $ b > a $ ist wahr)
    \item \texttt{16} ($ = b \times (a + 1) $, weil die erste Bedingung in der
        Fallunterscheidung \ttt{cond} falsch ist, w\"ahrend die zweite wahr ist)
\end{enumerate}
\end{loes}

\end{document}
